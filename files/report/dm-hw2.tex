%%%%%%%%%%%%%%%%%%%%%%%%%%%%%%%%%%%%%%%%%%%%%%%%%%%%%%%%%%%%%%
%                                                            %
%                    Read constant commands                  %
%                                                            %
%%%%%%%%%%%%%%%%%%%%%%%%%%%%%%%%%%%%%%%%%%%%%%%%%%%%%%%%%%%%%%

\input{../resources/constants/images path/IMAGE_PATH.tex}
\input{../resources/constants/images path/IMAGE_NAMES.tex}
\input{../resources/constants/images path/IMAGE_REFS.tex}
\input{../resources/constants/configuration/GENERAL_CONFIGURATION.tex}

%%%%%%%%%%%%%%%%%%%%%%%%%%%%%%%%%%%%%%%%%%%%%%%%%%%%%%%%%%%%%%
%                                                            %
%                       Custom Commands                      %
%                                                            %
%%%%%%%%%%%%%%%%%%%%%%%%%%%%%%%%%%%%%%%%%%%%%%%%%%%%%%%%%%%%%%

% Usage: \imageCaptionTable{image path}{image caption}{image label}{text}
% - The first argument should be the path to the image file in any format.
% - The second argument should be the caption of the image to be displayed.
% - The third argument should be the label of the image.
% - The fourth argument should be the text to be displayed on the right side of the image.
% - The resulting table will have the image displayed on the left (with the caption and label) and the text on the right.

\newcommand{\imageCaptionTable}[4]{%
	\begin{tabular}{l l}
		\includegraphics[width=0.5\textwidth, caption={#2}, label={#3}]{#1} & \centering
		\begin{varwidth}[t]{0.8\textwidth}
			\configuratedText{#4}
		\end{varwidth}
	\end{tabular}
}

% Inserts a small image in the middle of a text
% Usage: \smallimage[optional scale]{image file name}
% - The optional scale argument (default 0.5) can be used to adjust the size of the image.
% - The image file should be in the same directory as the LaTeX document.

\newcommand{\smallimage}[2][0.03]{
	\includegraphics[width=#1\linewidth]{#2}
}


% Creates a new figure with an image centered on the page, and adds a caption and label for reference.
% Usage: \sectionCenteredfigure[optional scale]{image file name}{caption text}{label}
% - The optional scale argument (default 0.9) can be used to adjust the width of the image.
% - The image file should be in the same directory as the LaTeX document.
% - The caption text should describe the contents of the image.
% - The label should be a unique identifier for the figure, used for referencing it later in the text.

\newcommand{\sectionCenteredfigure}[4][0.9]{
	\begin{figure}[H]
		\centering
		\fbox{\includegraphics[width=#1\linewidth]{#2}}
		\caption{#3}
		\label{fig:#4}
	\end{figure}
}


% Creates a new block of justified text with a specified font and font size.
% Usage: \generalText{font family}{font size}{text}
% - The font family argument specifies the font to be used (e.g., Times New Roman, Arial, etc.).
% - The font size argument specifies the size of the font (e.g., 12pt, 14pt, etc.).
% - The text argument should contain the text to be justified.
% - The resulting block of text will be fully justified (i.e., aligned with both the left and right margins).

\newcommand{\customText}[3]{%
	\par\begingroup
	\setlength{\parindent}{0pt}%
	\linespread{1.3}%
	\fontsize{#2}{#2}%
	\fontfamily{#1}\selectfont #3%
	\par\endgroup%
}

% Creates a new block of justified text with the font size and font style of configuration file.
% Usage: \generalText{text}
% - The font family argument is specfied by \userManualSimpleTextStyle 
% - The font size argument is specified by \userManualSimpleTextSize
% - The text argument should contain the text to be justified.
% - The resulting block of text will be fully justified (i.e., aligned with both the left and right margins).

\newcommand{\configuratedText}[1]{%
	\par\begingroup
	\setlength{\parindent}{0pt}%
	\linespread{1.3}%
	\selectfont #1%
	\par\endgroup%
}

% Defines a new command for referencing figures.
% Usage: \figref{label}
% - The argument should be the label of the figure to be referenced.
% - The resulting output will be in the format "(Figure <number>)", where <number> is the number of the referenced figure.
% - The label should be defined using \label{fig:<label>} command in the figure environment.
% - The \hyperref command creates a hyperlink to the referenced figure.
% - The \ref* command produces only the number of the referenced figure, without the preceding "Figure" text.

\newcommand{\figref}[1]{(\hyperref[fig:#1]{Figure \ref*{fig:#1}})}

% Creates a new table with an icon and its name.
% Usage: \customTable{icon path}{icon name}
% - The first argument should be the path to the icon file in PNG format.
% - The second argument should be the name of the icon to be displayed.
% - The resulting table will have the icon displayed on the left and its name on the right.

\newcommand{\iconNameTable}[2]{%
	\begin{tabular}{l l}
		\includegraphics[width=0.03\textwidth]{#1} & \centering 
		\begin{varwidth}[t]{0.8\textwidth}
			\configuratedText{#2}
		\end{varwidth}
	\end{tabular}
}

% Creates a new table with an icon and its name.
% Usage: \customTable{icon path}{icon name}
% - The first argument should be a text.
% - The second argument should be a text.
% - The resulting table will have the text displayed on the left and a text on the right.

\newcommand{\textTextTable}[3][2cm]{%
	\begin{tabular}{p{#1} p{\dimexpr0.90\textwidth-#1}}
		\configuratedText{#2}
		&
		\begin{varwidth}[t]{\linewidth}
			\configuratedText{#3}
		\end{varwidth}
	\end{tabular}
}

% Creates a new table with an icon, its name, and its description.
% Usage: \iconNameDescriptTable{icon path}{icon name}{icon description}
% - The first argument should be the path to the icon file in PNG format.
% - The second argument should be the name of the icon to be displayed.
% - The third argument should be a description of the icon.
% - The resulting table will have the icon displayed on the left, its name in the middle, and its description on the right.
% - The table has three columns with widths of 0.1, 0.3, and 0.5 times the text width, respectively.
% - The second and third columns are aligned to the left.

\newcommand{\iconNameDescriptTable}[3]{%
	\begin{tabular}{p{0.05\textwidth} p{0.2\textwidth} m{0.6\textwidth}}
		\includegraphics[width=0.03\textwidth]{#1} & \raggedright \configuratedText{#2} & \justify \configuratedText{#3} \
	\end{tabular}
}

% Creates a new table with an text, its name, and its description.
% Usage: \textDescriptTable{text}{text name}{text description}
% - The first argument should be the text to be displayed on the left.
% - The second argument should be the name of the text.
% - The third argument should be a description of the text.
% - The resulting table will have the text displayed on the left, its name in the middle, and its description on the right.
% - The table has three columns with widths of 0.1, 0.3, and 0.5 times the text width, respectively.
% - The second and third columns are aligned to the left.

\newcommand{\textDescriptTable}[3]{%
	\begin{tabular}{m{0.1\textwidth} m{0.1\textwidth} m{0.5\textwidth}}
		\raggedright \configuratedText{#1} & \raggedright \configuratedText{#2} & \raggedright \configuratedText{#3} \
	\end{tabular}
}

% Creates a new block of two columns with an image on the left and justified text on the right.
% Usage: \twocolumns{optional scale}{image file name}{caption text}{font family}{font size}{text}

% - 1: The optional scale argument (default 0.9) can be used to adjust the width of the image.
% - 2: The image file should be in the same directory as the LaTeX document.
% - 3: The caption text should describe the contents of the image.
% - 4: The font family argument specifies the font to be used (e.g., Times New Roman, Arial, etc.).
% - 5: The font size argument specifies the size of the font (e.g., 12pt, 14pt, etc.).
% - 6: The text argument should contain the text to be justified.

\newcommand{\twoColumns}[6]{
	\begin{minipage}[t]{0.50\textwidth}
		\customText{#4}{#5}{#6}
	\end{minipage}\hfill
	\begin{minipage}[r]{0.45\textwidth}

	\end{minipage}
}

\documentclass[12]{article}
\usepackage[utf8]{inputenc}
\usepackage{graphicx}
\usepackage{geometry}
\usepackage{tocloft}
\usepackage{amsmath}
\usepackage{booktabs}
\usepackage{fancyhdr}
\usepackage{hyperref}
\usepackage{xcolor}
\usepackage{soul}
\usepackage{times}
\usepackage{listings}
\usepackage{url}
\usepackage{wrapfig}
\usepackage{array}
\usepackage{varwidth}
\usepackage{float}
\usepackage{titlesec}
\usepackage{ragged2e}
\usepackage{todonotes}
\usepackage{tocloft}
\usepackage{changepage}
\usepackage{graphicx,scalerel}
\usepackage{amsmath}

\cftsetindents{section}{1.5em}{5.0em}
\cftsetindents{subsection}{2em}{5.0em}
\cftsetindents{subsubsection}{3em}{5.0em}

% Use the new command to set the font sizes and titleformat settings
\myheadingstyles

% Customize hyperlinks in the document
\hypersetup{
	colorlinks=true, % enable colored hyperlinks
	linkcolor=blue, % set the color of internal links to black
	filecolor=magenta, % set the color of links to local files to magenta
	urlcolor=blue, % set the color of links to URLs to blue
	bookmarks=true, % enable the creation of bookmarks in the PDF file
}

\geometry{
	a4paper, % set paper size to A4
	left=2cm, % set left margin to 2cm
	right=2cm, % set right margin to 2cm
	top=2.5cm, % set top margin to 2.5cm
	bottom=2.5cm % set bottom margin to 2.5cm
}


%%%%%%%%%%%%%%%%%%%%%%%%%%%%%%%%%%%%%%%%%%%%%%%%%%%%%%%%%%%%%%
%                                                            %
%                    Document Starts Here                    %
%                                                            %
%%%%%%%%%%%%%%%%%%%%%%%%%%%%%%%%%%%%%%%%%%%%%%%%%%%%%%%%%%%%%%

\begin{document}
	
	\begin{center}
		\begin{figure}[h]
			\centering
			\includegraphics[width=0.4\linewidth]{\dmFiveTwelve}
		\end{figure}
		\vspace{1cm} 
		{\fontsize{28}{34}\selectfont \textbf{Data Mining}}
	\end{center}

	\vspace{1cm} 
	
	\begin{center}
	{\fontsize{22}{28}\selectfont \textbf{Homework \underline{2}}}
	\end{center}

	\vspace{1cm} 

	\begin{center}
	{\fontsize{22}{28}\selectfont Ashkan Ansarifard}
	\end{center}

	\vspace{0.5cm} 
	
	\begin{center}
	{\fontsize{22}{28}\selectfont 1970082}
	\end{center}

	\vspace{1cm} 
	
	\textcolor{blue!60!black}{\rule{\linewidth}{2pt}}
	
	\vspace{5cm} 
	
	\begin{center}
	\textbf{A.Y. 2023/24}
	\end{center}

	\thispagestyle{empty}
	
	\newpage
	
	\myTOC

	\newpage
	\section*{Problem 1}\label{sec:prob-1}
	\configuratedText{In this report, I present the implementation and analysis of a Python program designed to scrape and analyze product data from Amazon using web scraping techniques. The objective of this problem is to collect information about products related to the keyword \textbf{\textit{gpu}} from the Amazon.it website, parse the data, and perform an Exploratory Data Analysis. \footnote{example usage class is \texttt{problem1\_main.py}}
	
	\subsection*{Web Scraping and Data Collection}
	The \texttt{AmazonScraper} class has been implemented to do the web scraping and data collection. The \texttt{scrape\_amazon\_products} method utilizes the \texttt{Requests} library to download web pages and \texttt{BeautifulSoup} for HTML parsing. It iterates through the specified number of pages, extracts relevant information for each product, and stores it in the \texttt{self.data} list.
	
	\subsection*{Tab-Separated Value (TSV) File}
	The \texttt{save\_to\_tsv} method converts the collected data into a Pandas DataFrame and saves it to a TSV file using \texttt{pd.to\_csv}. Each product's information is stored in a separate line, as written in the requirement of the problem statement.
	
	\subsection*{Delay to Prevent Blocking}
	To prevent being blocked by Amazon due to excessive requests, a delay of 5 seconds (\texttt{time.sleep(5)}) has been introduced between different web page requests.
	
	
	\subsection*{Exploratory Data Analysis (EDA)}
	The \texttt{analyze\_data} method has been implemented to perform an EDA on the collected dataset. The analysis includes\footnote{The EDA is from analyzing the first 10 pages}:
	
	\begin{itemize}
	    \item \textbf{Price Ranges}: Utilizing the Pandas \texttt{cut} function to categorize products into price ranges and visualizing the distribution using a box plot.
	    The dataset has been categorized into the following price ranges:
	    
	    \begin{table}[h]
	        \centering
	        \begin{tabular}{|c|c|}
	            \hline
	            \textbf{Price Range} & \textbf{Count} \\
	            \hline
	            500+ & 72 \\
	            $\le$100 & 33 \\
	            100-200 & 11 \\
	            200-300 & 4 \\
	            300-400 & 0 \\
	            400-500 & 0 \\
	            \hline
	        \end{tabular}
	        \caption{Distribution of Products Across Price Ranges (Console Output)}
	        \label{tab:price_ranges}
	    \end{table}
	    
	    \sectionCenteredfigure{\pricesDistribution}{Distribution of prices across different categories}{distribution-of-prices-across-different-categories}

	    \item \textbf{Customer Reviews}: Identifying and printing the top-rated products based on star ratings.
	    
	    \sectionCenteredfigure{\topRatedBasedOnStart}{Top Rated Products with Customer Reviews}{top-rated-products-with-customer-reviews}

	    \item \textbf{Primeness}: Separating the dataset into Prime and Non-Prime products and providing summary statistics for each category.
	    
	    \begin{table}[h]
	    \centering
	    \begin{tabular}{lccc}
	    \hline
	    & Price & Stars & Reviews \\
	    \hline
	    count & 31.000000 & 31.000000 & 31.000000 \\
	    mean & 3569.806452 & 1.458065 & 1409.645161 \\
	    std & 5977.225764 & 2.148142 & 2630.628120 \\
	    min & 39.000000 & 0.000000 & 0.000000 \\
	    25\% & 41.000000 & 0.000000 & 1.500000 \\
	    50\% & 147.000000 & 0.000000 & 5.000000 \\
	    75\% & 4195.000000 & 4.500000 & 2010.000000 \\
	    max & 20870.000000 & 4.700000 & 7114.000000 \\
	    \hline
	    \end{tabular}
	    \caption{Summary statistics for Prime Products. (Console Output)}
	    \label{table:prime-products}
	    \end{table}
	    
	    \begin{table}[h]
	    \centering
	    \begin{tabular}{lccc}
	    \hline
	    & Price & Stars & Reviews \\
	    \hline
	    count & 89.000000 & 89.000000 & 89.000000 \\
	    mean & 3292.707865 & 1.767416 & 526.483146 \\
	    std & 11187.100932 & 2.209112 & 1671.567141 \\
	    min & 7.000000 & 0.000000 & 0.000000 \\
	    25\% & 31.000000 & 0.000000 & 1.000000 \\
	    50\% & 132.000000 & 0.000000 & 7.000000 \\
	    75\% & 3195.000000 & 4.400000 & 62.000000 \\
	    max & 99990.000000 & 4.700000 & 7114.000000 \\
	    \hline
	    \end{tabular}
	    \caption{Summary statistics for Non-Prime Products. (Console Output)}
	    \label{table:non-prime-products}
	    \end{table}
	    
	    \item \textbf{Top 10 Products by Rating and Price}: Plotting the top 10 products based on both star rating and price using Plotly Express bar charts.
	    
	    \sectionCenteredfigure{\topProductsByPrice}{Top 10 Products by Price}{top-10-products-by-price}
	    \sectionCenteredfigure{\topProductsByStar}{Top 10 Products by Star Rating}{top-10-products-by-star-rating}

	    \item \textbf{Scatter Plot of Price vs. Star Rating}: Creating a scatter plot to visualize the relationship between product price and star rating, with marker size indicating the number of reviews.
	    
	    \sectionCenteredfigure{\scatterPriceStart}{Scatter Plot Price vs. Star Rating}{scatter-plot-price-vs}
	    \sectionCenteredfigure{\scatterStarReviews}{Scatter Plot Star Rating vs. Number of Reviews}{scatter-plot-star-rating-vs}

	\end{itemize}}
	
	\newpage
	\section*{Problem 2}\label{sec:prob-2}
	\configuratedText{In this section, I present the implementation of a search engine extension for the \texttt{AmazonScraper} class. The search engine focuses on evaluating queries based on the product's description, returning the top 10 most related products along with their information.\footnote{example usage class is \texttt{problem2\_main.py}}
		
		\subsection*{Class Implementation: AmazonScraperWithSearchEngine}
		The \texttt{AmazonScraperWithSearchEngine} class extends the \texttt{AmazonScraper} class to incorporate search engine functionalities. Below, I detail how each requirement in the problem statement has been fulfilled.
		
		\subsection*{Inverted Index and Cosine Similarity}
		The \texttt{evaluate\_query} method has been overridden to include vectorization and cosine similarity calculation. This method takes a user query, transforms it into a TF-IDF vector, and computes cosine similarities with the TF-IDF matrix of product descriptions. The related product indices are then sorted based on cosine similarities, and the top products are returned.
		
		\subsection*{Build Inverted Index}
		The \texttt{build\_inverted\_index} method initializes a TF-IDF vectorizer and computes the TF-IDF matrix based on the product descriptions. This serves as the inverted index for the search engine. It checks for the availability of data and prompts the user to run \texttt{scrape\_amazon\_products} if no data is present.
		
		\subsection*{Usage and Example}
		To use the search engine, first, instantiate the \texttt{AmazonScraperWithSearchEngine} class with a keyword and the number of pages to scrape. Run \texttt{scrape\_amazon\_products} to collect data, and then execute \texttt{build\_inverted\_index} to build the inverted index. Once the inverted index is available, queries can be evaluated using \texttt{evaluate\_query}.
		
		\subsubsection*{Example Usage}
		
		To demonstrate the usage of the \texttt{AmazonScraperWithSearchEngine} class, consider the following example which can be found also in \texttt{problem2\_main.py}:
		
		\begin{enumerate}
			\item \textbf{Instantiate the Search Engine:}
			\begin{quote}
				\texttt{search\_engine = AmazonScraperWithSearchEngine(keyword='gpu', num\_pages=5)}
			\end{quote}
			
			\item \textbf{Scrape Amazon Products:}
			\begin{quote}
				\texttt{search\_engine.scrape\_amazon\_products()}
			\end{quote}
			
			\item \textbf{Build the Inverted Index:}
			\begin{quote}
				\texttt{search\_engine.build\_inverted\_index(search\_engine.df['Description'])}
			\end{quote}
			
			\item \textbf{Evaluate a Query:}
			\begin{quote}
				\texttt{user\_query = "high-performance graphics card"} \\
				\texttt{query\_result = search\_engine.evaluate\_query(user\_query, top\_N=5)} \\
				\texttt{print("Query Result:")} \\
				\texttt{print(query\_result)}
			\end{quote}
		\end{enumerate}
		
		This example demonstrates the workflow of using the \textit{AmazonScraperWithSearchEngine} class. It begins by instantiating the class with a specified keyword and the number of pages to scrape. The class is then used to scrape Amazon products, build the inverted index based on product descriptions, and finally, evaluate a sample query using the cosine similarity measure. The resulting query result includes the top-related products along with their information.
		
		\subsubsection*{Example Usage Results}
		In the following, result of the search engine for different query lengths are shown. The list of queries are:
		\begin{enumerate}
			\item Gaming GPU
			
			\sectionCenteredfigure{\queryOne}{Query result - Gaming GPU}{gaming-gpu}
			
			\item GPU for video editing
			
			\sectionCenteredfigure{\queryTwo}{Query result - GPU for video editingd}{gpu-for-video-editing}
			
			\item Best budget GPU

			\sectionCenteredfigure{\queryThree}{Query result - Best budget GPU}{best-budget-gpu}
			
			\item NVIDIA GPU
			\sectionCenteredfigure{\queryFour}{Query result - NVIDIA GPU}{nvidia-gpu}
			
			\item AMD gaming graphics card
			
			\sectionCenteredfigure{\queryFive}{Query result - AMD gaming graphics card}{amd-gaming-graphics-card}
			
			\item Radeon RX 550 Low Profile Scheda Video, 4GB, GDDR5, 128-bit, VGA DVI-D HDMI, Video Card PC Gaming, 4k Displays, Computer GPU
		
			\sectionCenteredfigure{\queryLong}{Query result - Long Query GPU}{long-query-gpu}		
			
		\end{enumerate}
	}
	\newpage
	\section*{Problem 3}\label{sec:prob-3}
	\subsection*{Part 1: Shingling}
	The \texttt{Shingling} class generates shingles from a given document. Shingles are created by sliding a window of size $k$ through the document. Each shingle is a substring of length $k$. The \texttt{generate\_shingles} method creates a set of shingles from the input document.
	
	\subsection*{Part 2: Minwise Hashing}
	The \texttt{MinwiseHashSignature} class is designed for minwise hashing. It generates multiple hash functions using the MD5 algorithm. For each set of elements, the class updates a signature matrix with hash values. The \texttt{generate\_signatures} method takes a collection of sets and returns a signature matrix, where each column represents the minwise hash signature of a set.
	
	\subsection*{Part 3: Locality-Sensitive Hashing (LSH)}
	The \texttt{LSH} class implements the Locality-Sensitive Hashing technique. It uses minwise hash signatures to build hash tables. The \texttt{index\_signatures} method populates hash tables based on the signatures of the input sets. The \texttt{query\_signatures} method retrieves candidate sets by querying hash tables with the signature of a query set. The \texttt{find\_near\_duplicates} method uses LSH to find near-duplicates within a collection of shingles.
	
	\subsection*{Part 4: Threshold Intersection Analysis}
	The class contains a method named \texttt{s\_curve\_plot\_and\_analysis} that analyzes the threshold intersection for different values of $r$ and $b$. It iterates through combinations of $r$ and $b$, calculates the probability of becoming a candidate using the S-curve formula, and plots the results. The analysis involves checking for step-shaped curves that indicate optimal values for $r$ and $b$.
	
	\subsection*{Part 5: Choosing Optimal Parameters}
	The \texttt{choose\_r\_b\_values} method in the \texttt{LSH} class helps in selecting suitable values of $r$ and $b$ based on a given threshold probability. It iterates over possible combinations of $r$ and $b$, calculates the expected threshold probability, and selects values that closely match the given threshold.
	
	\subsection*{Part 6: Jaccard Similarity Calculation}
	The \texttt{ShingleComparison} class calculates Jaccard similarity between sets of shingles. The \texttt{compare\_shingle\_sets} method takes a collection of shingles and descriptions, computes Jaccard similarities, and identifies near-duplicate pairs based on a specified threshold.
	
	\subsection*{Performance Metrics}
	\begin{itemize}
	    \item The elapsed time for LSH execution is measured using the \texttt{elapsed\_time\_lsh} attribute.
	    \item The Jaccard similarities between shingle sets are stored in the \texttt{jaccard\_values} attribute.
	    \item The near-duplicate pairs and their Jaccard similarities are stored in a DataFrame (\texttt{df}) in the \texttt{ShingleComparison} class.
	\end{itemize}
	
	\newpage
	\section*{Problem 4}\label{sec:prob-4}
	The provided code implements the Locality-Sensitive Hashing technique using Apache Spark. The steps involve tokenizing product descriptions, creating Word2Vec representations, and using MinHashLSH for approximate similarity joins. Below is an overview of the key components of the Spark implementation:
	
	\begin{enumerate}
	    \item **Initialization:**
	    The Spark session is initialized with the name "AmazonScraperLSH".
	
	    \item **DataFrame Conversion:**
	    The Pandas DataFrame obtained from web scraping is converted to a Spark DataFrame (spark\_df).
	
	    \item **Tokenization:**
	    Product descriptions are tokenized using the `Tokenizer` class.
	
	    \item **Word2Vec Representation:**
	    Word2Vec representations of product descriptions are created using the `Word2Vec` model.
	
	    \item **MinHashLSH:**
	    MinHashLSH is applied to generate hash values for the features, and the DataFrame is divided into two halves for efficient processing.
	
	    \item **Approximate Similarity Join:**
	    The code performs an approximate similarity join using LSH with a specified Jaccard distance threshold.
	
	    \item **Result Presentation:**
	    The results are presented in the form of a DataFrame showing near-duplicate pairs along with their Jaccard distances.
	
	    \item **File Export:**
	    The final near-duplicate information is saved to a CSV file, and the file is made available for download.
	\end{enumerate}
\end{document}